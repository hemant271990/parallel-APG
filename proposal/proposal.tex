\documentclass{article}
\usepackage[nonatbib]{nips_2016}

\usepackage[breaklinks=true,letterpaper=true,colorlinks,citecolor=black,bookmarks=false]{hyperref}

\usepackage{amsthm}
\usepackage{amsmath,amssymb}
\usepackage{enumitem}

\usepackage[sort&compress,numbers]{natbib}
\usepackage[normalem]{ulem}

% use Times
\usepackage{times}
% For figures
\usepackage{graphicx} % more modern
%\usepackage{epsfig} % less modern
%\usepackage{subfig} 

\graphicspath{{../fig/}}

\usepackage{tikz}
\usepackage{tkz-tab}
\usepackage{caption} 
\usepackage{subcaption} 
\usetikzlibrary{shapes.geometric, arrows}
\tikzstyle{arrow} = [very thick,->,>=stealth]

\usepackage{cleveref}
\usepackage{setspace}
\usepackage{wrapfig}
%\usepackage[ruled]{algorithm}
\usepackage{algpseudocode}
\usepackage[noend,linesnumbered]{algorithm2e}

\usepackage[disable]{todonotes}


\title{Proposal for CS798, 2016 Fall\\ \large Optimization for Machine Learning}

\author{
	Hemant Saxena \\
	School of Computer Science\\
	University of Waterloo\\
	Waterloo, ON, N2L 3E6 \\
	\texttt{h2saxena@uwaterloo.ca} \\
	\And
	Royal Sequiera\\
	School of Computer Science\\
	University of Waterloo\\
	Waterloo, ON, N2L 3E6 \\
	\texttt{rdsequie@uwaterloo.ca} \\	
}

\begin{document}
\maketitle

\begin{abstract} 
The main idea of the project is to discuss the convergence and accuracy of the accelerated proximal gradient algorithm \cite{apg} over distributed computing framework.
The project will focus on two types of distributed computing frameworks: bulk synchronous parallel (BSP) systems, and stale sysnchronous parallel (SSP) system.
The convergence rate and the accuracy over the BSP and SSP systems will be compared against the single node implementation of the algorithm.
The implementation of the BSP version will be tested over Spark \cite{spark} and the SSP version will be tested over Petuum \cite{petuum}.
\end{abstract} 

\section{Introduction}
Due to the increasing volume of data most of the computation is being parallelized across multiple computing compute nodes.
This project will discuss the convergence and the accuracy of parallelizing one of the gradient descent algorithms, the accelerated proximal gradient(APG) algorithm.
APG algorithm is the optimal gradient descent algorithm with respect to convergance rate, which is $O(1/t^2)$ where $t$ is iteration count.
The parallel implementation of the algorithm will be tested over bulk synchronous parallel (BSP) systems, and stale sysnchronous parallel (SSP) system.
BSP systems enforces synchronization across worker nodes while progressing through intermediate stages, which makes the system slow due to straggelers.
On the other hand, SSP systems compromise consistency between workers and allow them to operate asynchronously, which makes the system fast but with bounded errors.

The project is motivated by the recent interest in parallelizing gradient descent algorithms.
In one of the recent works \cite{zhou2016convergence}, authors discussed the parallel implementation of proximal gradient algorithm.
Following that work, the next natural step would be to test the parallel implementation of the APG algorithm.

\section{Related Work}
The work on Stale Synchronous can be broadly divided into two major sets: 
(i) SSP systems, where individual machines skip updates while solving an optimization 
problem \cite{feyzmahdavian2014convergence,bertsekas1989convergence,bertsekas1989parallel,tseng1991rate,tsitsiklis1984distributed}. 
(ii) SSP systems, where machines do not allowed skip 
updates \cite{li2014scaling,li2013distributed,li2013parameter,feyzmahdavian2014delayed,ho2013more}.

Early research in this field, however, started in the late 
1980s \cite{bertsekas1989convergence,bertsekas1989parallel,tseng1991rate,tsitsiklis1984distributed}. 
Recently, Zhou et al. proposed \textbf{msPG}, 
an extension to the proximal gradient algorithm to the model parallel and stale synchronous setting\cite{zhou2016convergence}. 
The authors showed that \textbf{msPG} converges to a critical point under mild assumptions 
and such a critical point is optimal under convexity assumptions.

\section{Proposed Work}
We divide the work into three subtasks: first, implement parallel APG for Spark, second, implement parallel APG for Petuum, 
third, and compare the performance of both the implementations with respect to a single node implementation.
We will test the APG algorithm over a non-convex Lasso problem.
The data will be generated from $\mathcal{N}(0,1)$ withe normalized columns.
 
\section{Team}
Hemant: APG implementation for Spark.
Royal: APG implementation for Petuum.
Experiments will be designed and conducted jointly.

\section*{Acknowledgement}
We thank Professor Yaoliang Yu for his invaluable guidance, which led to the conception of this project.

\bibliographystyle{unsrtnat}
\bibliography{proposal}

\end{document}\grid
